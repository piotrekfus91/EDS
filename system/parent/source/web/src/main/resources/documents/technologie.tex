Branża informatyczna jest bardzo dynamicznie rozwijającą się obecnie branżą.
Na rynku istnieje szeroki wybór rónych technologii w każdej dziedzinie.
Możemy wybierać z produktów komercyjnych i niekomercyjncych, zapewniających wsparcie bądź nie.
Ponadto, technologie, które w jednej sytuacji wydają się być lepsze, w innej nie bądź spełniały oczekiwań.
Ten rozdział ma za zadanie zaprezentować najpopularniejsze dostępne na rynku technologie z rónych warstw.
Szczegłowiej zostaną opisane technologie używane w aplikacji.
\section{Systemy operacyjne}
System operacyjny to pierwsza decyzja, którą musimy podjąć opracowując aplikację.
W przeszłości, wybór ten był niezmiernie istotny.
Aplikacje nie były przenośne między systemami, więc aplikacja była pisana pod konkretny procesor i konkretny system operacyjny.
W dobie języka Java\footnote{http://www.oracle.com/technetwork/java/index.html} i aplikacji internetowych to wymaganie nie jest już tak istotne.
Dla aplikacji internetowych klientem jest przeglądarka, która jest dostępna na każdym systemie operacyjnym.
Z kolei serwery aplikacji oraz kontenery servletów są (w większości przypadków) przenośne i możemy je uruchomić na dowolnym systemie operacyjnym.
% TODO link do javy
Najważniejsze i najpopularniejsze systemy operacyjne zostały krótko opisane poniżej.
% TODO zrobic grafike statystyk SO
    \subsection{Microsoft Windows}
    Produkty z serii Microsoftu zajmują ponad 90\% rynku.
    Ich popularność wynika między innymi największej ilości dostępnych programów oraz z łatwości użytkowania.
    Na systemy Windows dostępne są zarówno przeglądarki internetowe pełniące role klientów jak i kontenery servletów pełniące rolę serwerów.
    \subsection{GNU/Linux}
    Zyskująca ostatnio coraz większą popularność rodzina systemów operacyjnych, mająca swe korzenie jeszcze w systemie UNIX (choć nie wywodząca się bezpośrednio).
    Systemy z tej rodziny kuszą stabilnością i bezpieczeństwem, dzięki dużo lepiej opracowanemu systemowi uprawnień, a także ceną - jako jedyne z popularniejszych systemów są darmowe.
    Wśród wielu implementacji Linuksów znajdziemy zarówno bardzo lekkie i szybkie systemy, jak i bardzo rozbudowane, w pełni wyposażone wersje.
    Do najpopularniejszych obecnie dystrybucji należą między innymi Ubuntu, Arch czy Fedora.
    Podobnie jak na systemy Windows, nie ma problemów z przeglądarkami czy serwerami aplikacji.
    \subsection{Mac OS}
    System rozprowadzany przez firmę Apple.
    Podobnie jak Linux swoje korzenie ma w systemie UNIX, ale jest jego potomkiem.
    Zaletami systemu są niewątpliwie ergonomia pracy i jego wygląd.
    Niestety, jest najdroższy z wymienionych tu systemów, co zniechęca ludzi do jego zakupu.
    Tu również nie ma problemów zarówno jako bycie klientem aplikacji jak i w roli serwera.
